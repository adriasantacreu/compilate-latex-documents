%%%%%%%%%%%%%%%%%%%%%%%%%%%%%%%%%%%%%%%%%%%%%%%%%%%%%%%%%%%%%%%%%%%%
% FITXA: ÀLGEBRA - EQUACIONS RACIONALS I FACTORITZADES
% Estil: Plantilla "Examen Modern" (Colors + Noto Sans + Logos)
%%%%%%%%%%%%%%%%%%%%%%%%%%%%%%%%%%%%%%%%%%%%%%%%%%%%%%%%%%%%%%%%%%%%

\documentclass[a4paper, 11pt]{article}

% --- PAQUETS ESSENCIALS ---
\usepackage[utf8]{inputenc}
\usepackage[catalan]{babel}
\usepackage{amsmath, amssymb}
\usepackage{graphicx}
\usepackage{tikz}
\usepackage{fontspec}
\usepackage[margin=2.5cm]{geometry} 
\usepackage{fancyhdr}
\usepackage{enumitem}
\usepackage{amsfonts}
\usepackage{xcolor}
\usepackage{soul}
\usepackage[most]{tcolorbox}
\usepackage{multicol}
\usepackage{array}
\usepackage{booktabs}
\usepackage{colortbl}
\usepackage{lastpage} 
\usepackage{pgfplots}
\usepackage{hyperref} 
\pgfplotsset{compat=1.18}
\usepgfplotslibrary{fillbetween}

% --- CONFIGURACIÓ DE FONTS ---
% Si no tens Noto Sans instal·lada, canvia per "Arial" o comenta aquestes línies
\setmainfont{Noto Sans}
\setsansfont{Noto Sans}
\newfontfamily{\titlefont}{Noto Sans} 
\renewcommand{\familydefault}{\sfdefault}

% --- CONFIGURACIÓ DE PARÀGRAFS (NO INDENTACIÓ) ---
\setlength{\parindent}{0pt} 
\setlength{\parskip}{0.5em} 

% --- COLORS I ESTILS ---
\definecolor{BrightPink}{HTML}{FF1493}
\definecolor{BrightBlue}{HTML}{00BFFF}
\definecolor{BrightGreen}{HTML}{32CD32}
\definecolor{BrightYellow}{HTML}{FFD700}
\definecolor{BrightPurple}{HTML}{9400D3}

% Llista de colors per als títols de les seccions
\def\questioncolors{{"BrightPurple","BrightGreen","BrightBlue","BrightPink"}}
\newcounter{qcolorindex}
\setcounter{qcolorindex}{0}
\newcommand{\currentqcolor}{}

\newcommand{\setsectioncolor}{%
  \stepcounter{qcolorindex}%
  \ifnum\theqcolorindex>4\setcounter{qcolorindex}{1}\fi%
  \pgfmathtruncatemacro{\myqindex}{\theqcolorindex-1}%
  \pgfmathparse{\questioncolors[\myqindex]}%
  \edef\currentqcolor{\pgfmathresult}%
}
\newcommand{\hlcurrent}[1]{{\sethlcolor{\currentqcolor!30!white}\hl{#1}}}

% Estils de caixes moderns
\tcbuselibrary{skins, breakable}
\newtcolorbox{infobox}[2][]{
    enhanced, breakable,
    title=#2, colback=black!5, colbacktitle=black!20,
    colframe=white, boxrule=0pt, arc=0mm, left=5mm,
    toptitle=1.5mm, bottomtitle=1.5mm, 
    overlay={%
        \draw[black!20, line width=2pt] 
        ([xshift=1pt]frame.south west) -- ([xshift=1pt]frame.north west);
    },
    fonttitle=\bfseries, coltitle=black, #1
}

% --- CONFIGURACIÓ ENLLAÇOS ---
\hypersetup{
    colorlinks=true,
    linkcolor=blue,
    filecolor=magenta,      
    urlcolor=BrightBlue,
}

% --- CONFIGURACIÓ CAPÇALERA/PEU ---
\pagestyle{fancy}
\fancyhf{}
\fancyfoot[R]{\tikz\node[circle, fill=black!70, text=white, font=\small\bfseries, inner sep=3pt]{\thepage};}
\renewcommand{\headrulewidth}{0pt}
\renewcommand{\footrulewidth}{0pt}
\setlength{\headheight}{25pt}

% --- LOGOS EN MARCA D'AIGUA (NOU) ---
\AddToHook{shipout/background}{%
    \begin{tikzpicture}[remember picture, overlay]
        % --- LOGO ESQUERRA ---
        \node[anchor=north west, xshift=0.5cm, yshift=-0.5cm, opacity=0.85] at (current page.north west) 
            { \includegraphics[height=1.4cm]{logo_institut.png} }; 
        
        % --- LOGO DRETA ---
        \node[anchor=north east, xshift=-0.5cm, yshift=-0.5cm, opacity=0.85] at (current page.north east) 
            { \includegraphics[height=1.3cm]{logo_generalitat.png} }; 
    \end{tikzpicture}%
}

% --- COMANDES PERSONALITZADES ---
\newcommand{\parttitle}[1]{%
  \setsectioncolor
  \par\vspace{1.5em}\noindent{\Large\titlefont\hlcurrent{#1}}\par\nopagebreak\vspace{0.8em}
}
\newcolumntype{C}[1]{>{\centering\arraybackslash}p{#1}}

\newcommand{\headersection}[2][Matemàtiques]{%
  \par\vspace*{0pt}
  \noindent \setsectioncolor
  {\Large\bfseries\color{\currentqcolor} #1}\\[0.1cm] 
  {\textbf{\color{black!70} Fitxa: \textit{#2}}}
    
  \vspace{0.2cm}
    
  \noindent
  \begin{tabular}{@{}p{10cm}p{5cm}@{}}
  \toprule
    \textbf{Nom i cognoms:} & \textbf{Data:} \\
    \bottomrule
  \end{tabular}
    
  \vspace{0.5cm}
}

% --- INICI DEL DOCUMENT ---
\begin{document}
\thispagestyle{fancy}

% --- CAPÇALERA ---
\headersection[Àlgebra]{Equacions Racionals i Factoritzades}

% --- INSTRUCCIONS ---
\begin{infobox}{Instruccions}
    Aquesta fitxa conté teoria en vídeo, exemples resolts i exercicis. Has de resoldre \textbf{almenys 4 exercicis} al teu quadern, indicant clarament el \textbf{MCM} en les equacions racionals.
\end{infobox}

% --- BLOC I: TEORIA ---
\parttitle{I. Teoria (Vídeos recomanats)}

Per resoldre aquests exercicis necessites dominar el Mínim Comú Múltiple de polinomis i la resolució d'equacions de grau superior.

\begin{enumerate}[label=\textbf{\arabic*.}, itemsep=0.8em]
    \item \textbf{Pas previ:} Com fer el MCM de polinomis.\\
    \href{https://www.youtube.com/watch?v=jCYUJ578S24}{\texttt{▶ Veure vídeo: Susi Profe - MCM de Polinomios}}
    
    \item \textbf{Tipus 1:} Com resoldre equacions amb fraccions algebraiques.\\
    \href{https://www.youtube.com/watch?v=P1Fjg1pdRQ0}{\texttt{▶ Veure vídeo: Susi Profe - Ecuaciones Racionales}}
    
    \item \textbf{Tipus 2:} Com resoldre equacions factoritzades (parèntesis).\\
    \href{https://www.youtube.com/watch?v=djG6T3fA5W4}{\texttt{▶ Veure vídeo: Susi Profe - Ecuaciones Factorizadas}}
\end{enumerate}

% --- BLOC II: EXEMPLES ---
\parttitle{II. Exemples Pràctics Resolts}

Analitza aquests dos exemples abans de començar els exercicis.

\begin{tcolorbox}[colback=white, colframe=black!10, title=\textbf{Exemple A: Equació Racional}]
    \textbf{Enunciat:} $\displaystyle \frac{2}{x} + \frac{x}{x+1} = 1$
    
    \textbf{1. Busquem el MCM:} Els denominadors són $x$, $(x+1)$ i $1$. 
    $$\text{MCM} = x(x+1)$$
    
    \textbf{2. Ajustem numeradors:}
    $$\frac{2(x+1)}{x(x+1)} + \frac{x(x)}{x(x+1)} = \frac{1 \cdot x(x+1)}{x(x+1)}$$
    
    \textbf{3. Eliminem denominadors i resolem:}
    \begin{align*}
        2(x+1) + x^2 &= x^2 + x \\
        2x + 2 + x^2 &= x^2 + x \quad \text{(Restem } x^2 \text{ als dos costats)} \\
        2x - x &= -2 \\
        \mathbf{x} &= \mathbf{-2}
    \end{align*}
    \textit{Comprovació: El denominador no es fa zero amb $-2$. Solució vàlida.}
\end{tcolorbox}

\vspace{0.5cm}

\begin{tcolorbox}[colback=white, colframe=black!10, title=\textbf{Exemple B: Equació Factoritzada}]
    \textbf{Enunciat:} $(x - 3) \cdot (x^2 - 4) = 0$
    
    \textbf{Idea clau:} Igualem cada factor (parèntesi) a zero per separat.
    
    \begin{itemize}
        \item Primer factor: $x - 3 = 0 \Rightarrow \mathbf{x = 3}$
        \item Segon factor: $x^2 - 4 = 0 \Rightarrow x^2 = 4 \Rightarrow x = \pm\sqrt{4} \Rightarrow \mathbf{x = 2, x = -2}$
    \end{itemize}
    
    \textbf{Solucions:} $x=3, x=2, x=-2$.
\end{tcolorbox}

\newpage

% --- BLOC III: EXERCICIS ---
\parttitle{III. Llistat d'Exercicis}

Resol els exercicis següents. Recorda indicar el MCM en les racionals.

\vspace{0.5cm}

\noindent\textbf{\color{\currentqcolor} Bloc A: Equacions Racionals}
\begin{enumerate}[label=\textbf{\arabic*.}, itemsep=1.2em]
    \item $\displaystyle \frac{1}{x} + \frac{1}{2} = 3$ 
    \hfill \textit{\small (Sol: $x = 2/5$)}

    \item $\displaystyle \frac{x}{3} + \frac{2}{x} = \frac{5}{3}$ 
    \hfill \textit{\small (Sol: $x = 2, x = 3$)}

    \item $\displaystyle \frac{5}{x-2} = \frac{3}{x+2}$ 
    \hfill \textit{\small (Sol: $x = -8$)}

    \item $\displaystyle \frac{2}{x+1} + \frac{3}{x-1} = \frac{4}{x^2-1}$ \quad \textit{\small (Pista: $x^2-1 = (x+1)(x-1)$)}
    \hfill \textit{\small (Sol: $x = 3/5$)}

    \item $\displaystyle \frac{x}{x-1} + 2x = 2$ 
    \hfill \textit{\small (Sol: $x = 1/2, x = 2$)}
\end{enumerate}

\vspace{1cm}

\noindent\textbf{\color{\currentqcolor} Bloc B: Equacions Factoritzades (Grau Superior)}
\begin{enumerate}[label=\textbf{\arabic*.}, start=6, itemsep=1.2em]
    \item $(x-5) \cdot (x+3) = 0$ 
    \hfill \textit{\small (Sol: $x=5, x=-3$)}

    \item $x \cdot (x-1) \cdot (x+4) = 0$ 
    \hfill \textit{\small (Sol: $x=0, x=1, x=-4$)}

    \item $(x^2 - 9) \cdot (x - 2) = 0$ 
    \hfill \textit{\small (Sol: $x=\pm 3, x=2$)}

    \item $(x^2 + 5) \cdot (2x - 10) = 0$ 
    \hfill \textit{\small (Sol: $x=5$)}

    \item $(x - 1)^2 \cdot (x + 2) = 0$ 
    \hfill \textit{\small (Sol: $x=1, x=-2$)}
\end{enumerate}

\vspace{1.5cm}
\hrule
\vspace{0.5cm}

\begin{infobox}[colback=yellow!10]{Recordatori important}
    En les equacions racionals, sempre has de \textbf{comprovar} que la solució obtinguda no faci zero cap denominador de l'equació original. Si un valor fa zero el denominador, cal descartar-lo!
\end{infobox}

\end{document}