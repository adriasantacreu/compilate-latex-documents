%%%%%%%%%%%%%%%%%%%%%%%%%%%%%%%%%%%%%%%%%%%%%%%%%%%%%%%%%%%%%%%%%%%%
% EXAMEN: MATEMÀTIQUES I - 1r BATXILLERAT
% Versió optimitzada per a GitHub Actions (XeLaTeX)
%%%%%%%%%%%%%%%%%%%%%%%%%%%%%%%%%%%%%%%%%%%%%%%%%%%%%%%%%%%%%%%%%%%%

\documentclass[a4paper, 11pt]{article}

% --- PAQUETS ESSENCIALS ---
\usepackage[catalan]{babel}
\usepackage{amsmath, amssymb, amsfonts}
\usepackage{graphicx}
\usepackage{tikz}
\usepackage{fontspec}
\usepackage[margin=2.5cm]{geometry} 
\usepackage{fancyhdr}
\usepackage{enumitem}
\usepackage{xcolor}
\usepackage{soul}
\usepackage[most]{tcolorbox}
\usepackage{multicol}
\usepackage{array}
\usepackage{booktabs}
\usepackage{colortbl}
\usepackage{lastpage} 

% --- CONFIGURACIÓ DE FONTS ---
% XeLaTeX buscarà aquestes fonts al sistema del servidor de GitHub
\setmainfont{Noto Sans}
\setsansfont{Noto Sans}
\renewcommand{\familydefault}{\sfdefault}

% --- COLORS I ESTILS ---
\definecolor{BrightPink}{HTML}{FF1493}
\definecolor{BrightBlue}{HTML}{00BFFF}
\definecolor{BrightGreen}{HTML}{32CD32}
\definecolor{BrightPurple}{HTML}{9400D3}

% Llista de colors per als títols
\def\questioncolors{{"BrightPurple","BrightGreen","BrightBlue"}}
\newcounter{qcolorindex}
\setcounter{qcolorindex}{0}
\newcommand{\currentqcolor}{}

\newcommand{\setsectioncolor}{%
  \stepcounter{qcolorindex}%
  \ifnum\theqcolorindex>3\setcounter{qcolorindex}{1}\fi%
  \pgfmathtruncatemacro{\myqindex}{\theqcolorindex-1}%
  \pgfmathparse{\questioncolors[\myqindex]}%
  \edef\currentqcolor{\pgfmathresult}%
}
\newcommand{\hlcurrent}[1]{{\sethlcolor{\currentqcolor!30!white}\hl{#1}}}

% --- CAIXES ---
\tcbuselibrary{skins, breakable}
\newtcolorbox{infobox}[2][]{
    enhanced, breakable,
    title=#2, colback=black!5, colbacktitle=black!20,
    colframe=white, boxrule=0pt, arc=0mm, left=5mm,
    toptitle=1.5mm, bottomtitle=1.5mm, 
    overlay={%
        \draw[black!20, line width=2pt] 
        ([xshift=1pt]frame.south west) -- ([xshift=1pt]frame.north west);
    },
    fonttitle=\bfseries, coltitle=black, #1
}

% --- LOGOS EN MARCA D'AIGUA ---
% NOTA: Si no puges els fitxers .png al repositori, comenta aquest bloc sencer amb '%'
\AddToHook{shipout/background}{%
    \begin{tikzpicture}[remember picture, overlay]
        % LOGO ESQUERRA
        % \node[anchor=north west, xshift=0.5cm, yshift=-0.5cm, opacity=0.85] at (current page.north west)
        %     { \includegraphics[height=1.4cm]{logo_institut.png} };
        
        % LOGO DRETA
        % \node[anchor=north east, xshift=-0.5cm, yshift=-0.5cm, opacity=0.85] at (current page.north east)
        %     { \includegraphics[height=1.3cm]{logo_generalitat.png} };
    \end{tikzpicture}%
}

% --- CAPÇALERA I PEU ---
\pagestyle{fancy}
\fancyhf{}
\fancyfoot[R]{\tikz\node[circle, fill=black!70, text=white, font=\small\bfseries, inner sep=3pt]{\thepage};}
\renewcommand{\headrulewidth}{0pt}
\setlength{\headheight}{25pt}

% --- COMANDES PERSONALITZADES ---
\newcommand{\parttitle}[1]{%
  \setsectioncolor
  \par\vspace{1.5em}\noindent{\Large\hlcurrent{#1}}\par\nopagebreak\vspace{0.8em}
}
\newcolumntype{C}[1]{>{\centering\arraybackslash}p{#1}}

\newcommand{\headersection}[2][Matemàtiques I]{%
  \par\vspace*{0pt}
  \noindent \setsectioncolor
  {\Large\bfseries\color{\currentqcolor} #1}\\[0.1cm] 
  {\textbf{\color{black!70} Unitat: \textit{#2}}}
  \par\vspace{0.4cm} 
  \noindent
  \begin{tabular}{@{}p{10cm}p{5cm}@{}}
  \toprule
    \textbf{Nom i cognoms:} \hspace{0.25cm} & \textbf{Qualificació:} \\
    \bottomrule
  \end{tabular}
  \par
  \vspace{0.5cm}
}

% COMANDA PER A PÀGINA EN BLANC (CANVIADA PER EVITAR ERROR)
\newcommand{\fullenblanc}{
    \clearpage
    \thispagestyle{empty}
    \null
    \clearpage
}

% --- INICI DEL DOCUMENT ---
\begin{document}
\thispagestyle{fancy}

% --- CAPÇALERA ---
\headersection[Matemàtiques I]{Recuperació 1a Avaluació}

% --- INSTRUCCIONS ---
\begin{infobox}{Instruccions Generals}
    \textbf{Temps:} 90 minuts. \textbf{Calculadora:} NO permesa. \par\vspace{0.5em}
    \textbf{Part A (Test - 4 punts):} 15 preguntes. Respon al full de respostes. \par
    \textit{Puntuació:} Encerts: +0,26 | Errors: -0,08 | En blanc: 0. \par\vspace{0.5em}
    \textbf{Part B (Problemes - 6 punts):} 4 exercicis de desenvolupament.
\end{infobox}

% --- PART A: TEST ---
\parttitle{Part A: Qüestionari Tipus Test (4 punts)}

\begin{enumerate}[label=\textbf{\arabic*.}, itemsep=0.4em]
    \item Quin dels següents nombres pertany al conjunt dels Irracionals ($\mathbb{I}$)?
    \begin{multicols}{4}
    \begin{enumerate}[label=\alph*)]
        \item $2,71$
        \item $\sqrt{16}$
        \item $1,333...$
        \item $1-\sqrt{3}$
    \end{enumerate}
    \end{multicols}

    \item La intersecció dels intervals $A = [-3, 5)$ i $B = (0, +\infty)$ és:
    \begin{multicols}{4}
    \begin{enumerate}[label=\alph*)]
        \item $[-3, 0)$
        \item $(0, 5)$
        \item $(0, 5]$
        \item $[-3, +\infty)$
    \end{enumerate}
    \end{multicols}
\end{enumerate}

\newpage

% --- FULL DE RESPOSTES ---
\headersection[Matemàtiques I]{Recuperació}
\parttitle{Full de respostes (Part A)}
\vspace{1em}
\noindent\textit{Marca amb una X la casella correcta.}
\vspace{1em}

\renewcommand{\arraystretch}{2.2}
\begin{center}
\begin{tabular}{|*{5}{C{1.2cm}|}}
\hline
\rowcolor{black!15} \textbf{1} & \textbf{2} & \textbf{3} & \textbf{4} & \textbf{5} \\
\hline
 & & & & \\
\hline
\end{tabular}
\end{center}

\fullenblanc % Exemple d'ús de la nova comanda de pàgina en blanc

% --- PART B: PROBLEMES ---
\headersection[Matemàtiques I]{Recuperació}
\parttitle{Part B: Problemes (6 punts)}

\noindent{\large\bfseries Problema 1 (1,5 punts): Operacions amb Radicals}
\par\vspace{1em}
$$3\sqrt{8} - 5\sqrt{18} + \sqrt{50}$$

\end{document}