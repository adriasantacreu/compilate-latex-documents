%%%%%%%%%%%%%%%%%%%%%%%%%%%%%%%%%%%%%%%%%%%%%%%%%%%%%%%%%%%%%%%%%%%%
% EXAMEN: MATEMÀTIQUES I - 1r BATXILLERAT
% Recuperació 1a Avaluació
% Estructura: 15 Test + 4 Problemes (amb fulls en blanc i LOGOS)
% Correcció: Salt de línia a la capçalera arreglat.
%%%%%%%%%%%%%%%%%%%%%%%%%%%%%%%%%%%%%%%%%%%%%%%%%%%%%%%%%%%%%%%%%%%%

\documentclass[a4paper, 11pt]{article}

% --- PAQUETS ESSENCIALS ---
\usepackage[utf8]{inputenc}
\usepackage[catalan]{babel}
\usepackage{amsmath, amssymb}
\usepackage{graphicx}
\usepackage{tikz}
\usepackage{fontspec}
\usepackage[margin=2.5cm]{geometry} 
\usepackage{fancyhdr}
\usepackage{enumitem}
\usepackage{amsfonts}
\usepackage{xcolor}
\usepackage{soul}
\usepackage[most]{tcolorbox}
\usepackage{multicol}
\usepackage{array}
\usepackage{booktabs}
\usepackage{colortbl}
\usepackage{lastpage} 

% --- CONFIGURACIÓ DE FONTS ---
\setmainfont{Noto Sans}
\setsansfont{Noto Sans}
\newfontfamily{\titlefont}{Noto Sans} 
\renewcommand{\familydefault}{\sfdefault}

% --- COLORS I ESTILS ---
\definecolor{BrightPink}{HTML}{FF1493}
\definecolor{BrightBlue}{HTML}{00BFFF}
\definecolor{BrightGreen}{HTML}{32CD32}
\definecolor{BrightPurple}{HTML}{9400D3}

% Llista de colors per als títols
\def\questioncolors{{"BrightPurple","BrightGreen","BrightBlue"}}
\newcounter{qcolorindex}
\setcounter{qcolorindex}{0}
\newcommand{\currentqcolor}{}

\newcommand{\setsectioncolor}{%
  \stepcounter{qcolorindex}%
  \ifnum\theqcolorindex>3\setcounter{qcolorindex}{1}\fi%
  \pgfmathtruncatemacro{\myqindex}{\theqcolorindex-1}%
  \pgfmathparse{\questioncolors[\myqindex]}%
  \edef\currentqcolor{\pgfmathresult}%
}
\newcommand{\hlcurrent}[1]{{\sethlcolor{\currentqcolor!30!white}\hl{#1}}}

% --- CAIXES ---
\tcbuselibrary{skins, breakable}
\newtcolorbox{infobox}[2][]{
    enhanced, breakable,
    title=#2, colback=black!5, colbacktitle=black!20,
    colframe=white, boxrule=0pt, arc=0mm, left=5mm,
    toptitle=1.5mm, bottomtitle=1.5mm, 
    overlay={%
        \draw[black!20, line width=2pt] 
        ([xshift=1pt]frame.south west) -- ([xshift=1pt]frame.north west);
    },
    fonttitle=\bfseries, coltitle=black, #1
}

% --- LOGOS EN MARCA D'AIGUA (INTEGRATS) ---
\AddToHook{shipout/background}{%
    \begin{tikzpicture}[remember picture, overlay]
        
        % --- LOGO ESQUERRA ---
        \node[anchor=north west, xshift=0.5cm, yshift=-0.5cm, opacity=0.85] at (current page.north west)
            { \includegraphics[height=1.4cm]{logo_institut.png} };
        
        % --- LOGO DRETA ---
        \node[anchor=north east, xshift=-0.5cm, yshift=-0.5cm, opacity=0.85] at (current page.north east)
            { \includegraphics[height=1.3cm]{logo_generalitat.png} };
            
    \end{tikzpicture}%
}

% --- CAPÇALERA I PEU ---
\pagestyle{fancy}
\fancyhf{}
\fancyfoot[R]{\tikz\node[circle, fill=black!70, text=white, font=\small\bfseries, inner sep=3pt]{\thepage};}
\renewcommand{\headrulewidth}{0pt}
\setlength{\headheight}{25pt}

% --- COMANDES ---
\newcommand{\parttitle}[1]{%
  \setsectioncolor
  \par\vspace{1.5em}\noindent{\Large\titlefont\hlcurrent{#1}}\par\nopagebreak\vspace{0.8em}
}
\newcolumntype{C}[1]{>{\centering\arraybackslash}p{#1}}

% --- CAPÇALERA CORREGIDA ---
\newcommand{\headersection}[2][Matemàtiques I]{%
  \par\vspace*{0pt}
  \noindent \setsectioncolor
  {\Large\bfseries\color{\currentqcolor} #1}\\[0.1cm] 
  {\textbf{\color{black!70} Unitat: \textit{#2}}}
  
  \par\vspace{0.4cm} % SALT DE LÍNIA AFEGIT AQUÍ
  
  \noindent
  \begin{tabular}{@{}p{10cm}p{5cm}@{}}
  \toprule
    \textbf{Nom i cognoms:} \hspace{0.25cm} {\Large \%NOM\%} & \textbf{Qualificació:} \\
    \bottomrule
  \end{tabular}
  \par
  \vspace{0.5cm}
}

% Comanda especial per crear pàgina en blanc
\newcommand{\newpage}{
    \newpage
    \thispagestyle{empty}
    \null
    \newpage
}

% --- INICI DEL DOCUMENT ---
\begin{document}
\thispagestyle{fancy}

% --- CAPÇALERA ---
\headersection[Matemàtiques I]{Recuperació 1a Avaluació}

% --- INSTRUCCIONS ---
\begin{infobox}{Instruccions Generals}
    \textbf{Temps:} 90 minuts. \textbf{Calculadora:} NO permesa. \par\vspace{0.5em}
    \textbf{Part A (Test - 4 punts):} 15 preguntes. Respon al full de respostes. \par
    \textit{Puntuació:} Encerts: +0,26 | Errors: -0,08 | En blanc: 0. \par\vspace{0.5em}
    \textbf{Part B (Problemes - 6 punts):} 4 exercicis de desenvolupament (1,5 punts cadascun). Cal justificar tots els passos.
\end{infobox}

% --- PART A: TEST ---
\parttitle{Part A: Qüestionari Tipus Test (4 punts)}

\begin{enumerate}[label=\textbf{\arabic*.}, itemsep=0.4em]

    % P1
    \item Quin dels següents nombres pertany al conjunt dels Irracionals ($\mathbb{I}$)?
    \begin{multicols}{4}
    \begin{enumerate}[label=\alph*)]
        \item $2,71$
        \item $\sqrt{16}$
        \item $1,333...$
        \item $1-\sqrt{3}$
    \end{enumerate}
    \end{multicols}

    % P2
    \item La intersecció dels intervals $A = [-3, 5)$ i $B = (0, +\infty)$ és:
    \begin{multicols}{4}
    \begin{enumerate}[label=\alph*)]
        \item $[-3, 0)$
        \item $(0, 5)$
        \item $(0, 5]$
        \item $[-3, +\infty)$
    \end{enumerate}
    \end{multicols}

    % P3
    \item Si racionalitzem la fracció $\frac{3}{\sqrt{3}}$, quin és el resultat simplificat?
    \begin{multicols}{4}
    \begin{enumerate}[label=\alph*)]
        \item $3$
        \item $1$
        \item $\sqrt{3}$
        \item $3\sqrt{3}$
    \end{enumerate}
    \end{multicols}

    % P4
    \item Quin és el resultat de $(2 \cdot 10^{-4}) \cdot (3 \cdot 10^7)$ en notació científica?
    \begin{multicols}{4}
    \begin{enumerate}[label=\alph*)]
        \item $6 \cdot 10^{-28}$
        \item $5 \cdot 10^3$
        \item $6 \cdot 10^{11}$
        \item $6 \cdot 10^3$
    \end{enumerate}
    \end{multicols}

    % P5
    \item L'expressió $\sqrt[5]{a^2 \cdot \sqrt{a}}$ és equivalent a:
    \begin{multicols}{4}
    \begin{enumerate}[label=\alph*)]
        \item $a^{1/2}$
        \item $a^{2/5}$
        \item $a^{1/5}$
        \item $a^{5/2}$
    \end{enumerate}
    \end{multicols}

    % P6
    \item Si $\log_x(25) = 2$, quin és el valor de $x$?
    \begin{multicols}{4}
    \begin{enumerate}[label=\alph*)]
        \item 12,5
        \item 5
        \item 10
        \item 625
    \end{enumerate}
    \end{multicols}

    % P7
    \item Aplicant les propietats, a què és igual $\ln(e^3) + \log(100)$?
    \begin{multicols}{4}
    \begin{enumerate}[label=\alph*)]
        \item 5
        \item $e^3 + 100$
        \item 13
        \item $2e + 2$
    \end{enumerate}
    \end{multicols}

    \newpage

    % P8
    \item L'expressió $2\log(A) - \log(B)$ es condensa en:
    \begin{multicols}{4}
    \begin{enumerate}[label=\alph*)]
        \item $\log(\frac{2A}{B})$
        \item $\log(A^2 - B)$
        \item $\log(\frac{A^2}{B})$
        \item $\frac{\log(A^2)}{\log(B)}$
    \end{enumerate}
    \end{multicols}


    % P9
    \item Sabent que $\log(2) \approx 0,3$, quant val $\log(20)$?
    \begin{multicols}{4}
    \begin{enumerate}[label=\alph*)]
        \item 2,3
        \item 1,3
        \item 0,6
        \item 20,3
    \end{enumerate}
    \end{multicols}

    % P10
    \item Quin és el grau del polinomi $P(x) = (x^2 + 1)^3 - x^6$?
    \begin{multicols}{4}
    \begin{enumerate}[label=\alph*)]
        \item Grau 6
        \item Grau 5
        \item Grau 4
        \item Grau 0
    \end{enumerate}
    \end{multicols}

    % P11
    \item Segons el Teorema del Residu, el residu de dividir $P(x)$ entre $(x+2)$ és:
    \begin{multicols}{4}
    \begin{enumerate}[label=\alph*)]
        \item $P(2)$
        \item $P(-2)$
        \item $-2$
        \item $0$
    \end{enumerate}
    \end{multicols}

    % P12
    \item Si un polinomi $P(x)$ té arrels $1, -1, 0$, la factorització podria ser:
    \begin{multicols}{2}
    \begin{enumerate}[label=\alph*)]
        \item $(x-1)(x+1)$
        \item $(x+1)^2(x-1)$
        \item $x(x-1)(x+1)$
        \item $x(x^2+1)$
    \end{enumerate}
    \end{multicols}

    % P13
    \item Quin terme falta a la identitat $(2x - 3)^2 = 4x^2 - \dots + 9$?
    \begin{multicols}{4}
    \begin{enumerate}[label=\alph*)]
        \item $6x$
        \item $6$
        \item $12x^2$
        \item $12x$
    \end{enumerate}
    \end{multicols}

    % P14
    \item Si el residu d'una divisió per Ruffini és 0, podem afirmar que:
    \begin{enumerate}[label=\alph*)]
        \item La divisió és exacta i el divisor és factor del dividend.
        \item El quocient és 0.
        \item El polinomi no té arrels reals.
        \item El dividend és un nombre constant.
    \end{enumerate}

    % P15
    \item La solució de la inequació $|x| < 3$ és l'interval:
    \begin{multicols}{4}
    \begin{enumerate}[label=\alph*)]
        \item $[-3, 3]$
        \item $(-\infty, 3)$
        \item $(3, +\infty)$
        \item $(-3, 3)$
    \end{enumerate}
    \end{multicols}

\end{enumerate}

\newpage

% --- FULL DE RESPOSTES ---
\headersection[Matemàtiques I]{Recuperació}

\parttitle{Full de respostes (Part A)}
\vspace{1em}
\noindent\textit{Marca amb una X la casella correcta.}
\vspace{1em}

\renewcommand{\arraystretch}{2.2}
\begin{center}
\begin{tabular}{|*{5}{C{1.5cm}|}}
\hline
\rowcolor{black!15} \textbf{1} & \textbf{2} & \textbf{3} & \textbf{4} & \textbf{5} \\
\hline
 & & & & \\
\hline
\end{tabular}
\vspace{0.5cm}

\begin{tabular}{|*{5}{C{1.5cm}|}}
\hline
\rowcolor{black!15} \textbf{6} & \textbf{7} & \textbf{8} & \textbf{9} & \textbf{10} \\
\hline
 & & & & \\
\hline
\end{tabular}
\vspace{0.5cm}

\begin{tabular}{|*{5}{C{1.5cm}|}}
\hline
\rowcolor{black!15} \textbf{11} & \textbf{12} & \textbf{13} & \textbf{14} & \textbf{15} \\
\hline
 & & & & \\
\hline
\end{tabular}
\end{center}

\newpage
\null
\newpage

% --- PART B: PROBLEMES ---
\headersection[Matemàtiques I]{Recuperació}
\parttitle{Part B: Problemes (6 punts)}

% PROBLEMA 1
\noindent{\large\bfseries Problema 1 (1,5 punts): Operacions amb Radicals}
\par\vspace{1em}
\begin{enumerate}[label=\alph*)]
    \item Opera i simplifica al màxim la següent expressió, extraient tots els factors possibles de les arrels:
    $$3\sqrt{8} - 5\sqrt{18} + \sqrt{50}$$
    \item Racionalitza el denominador i simplifica el resultat:
    $$\frac{4}{\sqrt{5} - 1}$$
\end{enumerate}

\newpage
\null
\newpage

% PROBLEMA 2
\headersection[Matemàtiques I]{Recuperació}
\noindent{\large\bfseries Problema 2 (1,5 punts): Equacions Logarítmiques}
\par\vspace{1em}
\begin{enumerate}[label=\alph*)]
    \item Resol la següent equació logarítmica pas a pas, utilitzant les propietats:
    $$\log(x) + \log(x - 9) = 1$$
    (Recorda comprovar la validesa de les solucions trobades).
    \item Calcula el valor de $x$ a la següent expressió sense usar calculadora:
    $$\log_3(\sqrt{27}) - \log_2\left(\frac{1}{4}\right) = x$$
\end{enumerate}

\newpage
\null
\newpage

% PROBLEMA 3
\headersection[Matemàtiques I]{Recuperació}
\noindent{\large\bfseries Problema 3 (1,5 punts): Polinomis i Paràmetres}
\par\vspace{1em}
\noindent Considera el polinomi $P(x) = x^3 + mx^2 + nx - 6$.
\begin{enumerate}[label=\alph*)]
    \item Determina els valors dels paràmetres $m$ i $n$ sabent que:
    \begin{itemize}
        \item El polinomi és divisible per $(x - 1)$.
        \item El residu de dividir el polinomi per $(x + 2)$ és $-12$.
    \end{itemize}
    \item Un cop trobats $m$ i $n$, factoritza completament el polinomi i indica quines són totes les seves arrels.
\end{enumerate}

\newpage
\null
\newpage

% PROBLEMA 4
\headersection[Matemàtiques I]{Recuperació}
\noindent{\large\bfseries Problema 4 (1,5 punts): Valor Absolut}
\par\vspace{1em}
\noindent Donada la funció $f(x) = |x + 1| + |x - 4|$:
\begin{enumerate}[label=\alph*)]
    \item Expressa $f(x)$ com una funció definida a trossos, eliminant els valors absoluts. Has d'estudiar els signes en els intervals corresponents.
    \item Calcula el valor de la funció per a $x = -2$ i $x = 2$ utilitzant l'expressió a trossos que has trobat a l'apartat anterior.
\end{enumerate}

\newpage
\null
\newpage

\end{document}